\documentclass{scrartcl}

% Paket float verbessern
\usepackage{scrhack}

% Kleine Anpassung von maketitle
\usepackage{titling}
% \usepackage{showframe}

% Warnung, falls nochmal kompiliert werden muss
\usepackage[aux]{rerunfilecheck}

% unverzichtbare Mathe-Befehle
\usepackage{amsmath}
% viele Mathe-Symbole
\usepackage{amssymb}
% Erweiterungen für amsmath
\usepackage{mathtools}

% Fonteinstellungen
\usepackage{selnolig}
\usepackage{fontspec}
% Latin Modern Fonts werden automatisch geladen
% Alternativ zum Beispiel:
%\setromanfont{Libertinus Serif}
%\setsansfont{Libertinus Sans}
%\setmonofont{Libertinus Mono}

% Wenn man andere Schriftarten gesetzt hat,
% sollte man das Seiten-Layout neu berechnen lassen
\recalctypearea{}

% deutsche Spracheinstellungen
\usepackage{polyglossia}
\setmainlanguage{german}
\setotherlanguage{english}

\usepackage[
  math-style=ISO,    % ┐
  bold-style=ISO,    % │
  sans-style=italic, % │ ISO-Standard folgen
  nabla=upright,     % │
  partial=upright,   % ┘
  warnings-off={           % ┐
    mathtools-colon,       % │ unnötige Warnungen ausschalten
    mathtools-overbracket, % │
  },                       % ┘
]{unicode-math}

% traditionelle Fonts für Mathematik
\setmathfont{Latin Modern Math}
% Alternativ zum Beispiel:
%\setmathfont{Libertinus Math}

\setmathfont{XITS Math}[range={scr, bfscr}]
\setmathfont{XITS Math}[range={cal, bfcal}, StylisticSet=1]

% Zahlen und Einheiten
\usepackage[
  locale=DE,                   % deutsche Einstellungen
  separate-uncertainty=true,   % immer Fehler mit \pm
  per-mode=symbol-or-fraction, % / in inline math, fraction in display math
]{siunitx}

% chemische Formeln
\usepackage[
  version=4,
  math-greek=default, % ┐ mit unicode-math zusammenarbeiten
  text-greek=default, % ┘
]{mhchem}

% richtige Anführungszeichen
\usepackage[autostyle]{csquotes}

% schöne Brüche im Text
\usepackage{xfrac}

% Standardplatzierung für Floats einstellen
\usepackage{float}
\floatplacement{figure}{htbp}
\floatplacement{table}{htbp}

% Floats innerhalb einer Section halten
\usepackage[
  section, % Floats innerhalb der Section halten
  below,   % unterhalb der Section aber auf der selben Seite ist ok
]{placeins}

% Seite drehen für breite Tabellen: landscape Umgebung
\usepackage{pdflscape}

% Captions schöner machen.
\usepackage[
  labelfont=bf,        % Tabelle x: Abbildung y: ist jetzt fett
  font=small,          % Schrift etwas kleiner als Dokument
  width=0.9\textwidth, % maximale Breite einer Caption schmaler
]{caption}
% subfigure, subtable, subref
\usepackage{subcaption}

% Grafiken können eingebunden werden
\usepackage{graphicx}
% größere Variation von Dateinamen möglich
\usepackage{grffile}

% schöne Tabellen
\usepackage{booktabs}

% Verbesserungen am Schriftbild
\usepackage{microtype}

% Literaturverzeichnis
\usepackage[
  backend=biber,
]{biblatex}
% Quellendatenbank
% \addbibresource{lit.bib}
% \addbibresource{programme.bib}

% Hyperlinks im Dokument
\usepackage[
  unicode,        % Unicode in PDF-Attributen erlauben
  pdfusetitle,    % Titel, Autoren und Datum als PDF-Attribute
  pdfcreator={},  % ┐ PDF-Attribute säubern
  pdfproducer={}, % ┘
]{hyperref}
% erweiterte Bookmarks im PDF
\usepackage{bookmark}
\usepackage{cleveref} % muss nach hyperref kommen

% Trennung von Wörtern mit Strichen
\usepackage[shortcuts]{extdash}

\usepackage{expl3}
\usepackage{xparse}

\usepackage{minted}
\usepackage{enumerate}
\usepackage{exsheets}
\SetupExSheets{
  headings=block-subtitle,
  question/name=Aufgabe,
}

\DeclareInstance{exsheets-heading}{block-subtitle}{default}{
  join = {
    title[r,B]number[l,B](.333em,0pt) ;
    title[r,B]subtitle[l,B](1em,0pt)
  } ,
  attach = {
    main[l,vc]title[l,vc](0pt,0pt) ;
    main[r,vc]points[l,vc](\marginparsep,0pt)
  }
}

\setlength{\droptitle}{-4em}
\pretitle{\begin{center}\Large}
\posttitle{\par\end{center}}
\preauthor{
  \begin{center}\large \lineskip 0.2em
    \begin{tabular}[t]{c}
}
\postauthor{\end{tabular}\par\end{center}}
\predate{\begin{center}\normalsize}
\postdate{\par\end{center}}

\setlength{\parindent}{0cm}
\setlength{\topmargin}{0in}
\setlength{\headheight}{0cm}
\setlength{\headsep}{0cm}
\setlength{\textheight}{9.0in}
\setlength{\evensidemargin}{0.0in}
\setlength{\oddsidemargin}{0.0in}
\setlength{\textwidth}{6in}
\setlength{\footskip}{.8in}

\title{\textsf{\textbf{Übungen zum wissenschaftlichen Programmieren SS2019}}}
\author{Prof.\,Dr.\,W.\,Kilian\unskip, S.\,Braß}

\ExplSyntaxOn
\RenewDocumentCommand \maketitlehooka {} {
  \rule{\textwidth}{1pt}
}
\RenewDocumentCommand \maketitlehookd {} {
  \rule{\textwidth}{1pt}
}
\RenewDocumentCommand \maketitlehookc {} {
  \begin{center}
    \large \AufgabenBlatt
  \end{center}
}
\NewDocumentCommand \I {}
{
\symup{i}
}
\NewDocumentCommand \Sp {}
{
  \operatorname{Sp}
}
\NewDocumentCommand \adj {}
{
  \operatorname{adj}
}
\ExplSyntaxOff

\NewDocumentCommand\AufgabenBlatt{}{Hausarbeit}
\date{Ausgabe: Fr, 12.07.2019, Abgabe: Do, 18.07.2019, 18:00 Uhr}
\setcounter{question}{0}
\begin{document}

\maketitle

% Betrachten Sie eine gewöhnliche Differentialgleichung (DGL) $n$-ter Ordnung für eine (vektorwertige) Funktion $\vec{y} = \vec{y}(x)$ in einer Variable $x$, welche in der allgemeinsten Form gegeben ist als
% \begin{equation}
%   \label{eq:1}
%   \vec{y}^{(n)} = f(x, \vec{y}, \vec{y}^\prime, \vec{y}^{\prime\prime}, …, \vec{y}^{(n - 1)}).
% \end{equation}
% Die allgemeine DGL in~\eqref{eq:1} kann zu einem Gleichungssystem von $n$ Differentialgleichungen 1.~Ordnung reduziert werden, wobei Sie mit
% \begin{equation}
%   \label{eq:2}
%   \symbf{y} ≡ \begin{pmatrix}
%     \vec{y}₁ = & \vec{y}\\
%     \vec{y}₂ = & \vec{y}^\prime\\
%     \vdots & \\
%     \vec{y}ₙ = & \vec{y}^{(n - 1)}
%   \end{pmatrix},
% \end{equation}
% eine Differentialgleichung 1. Ordnung für den Funktionen-Vektor $\symbf{y}(x)$ erhalten,
% \begin{equation}
%   \label{eq:3}
%   \symbf{y}^{\prime} = \begin{pmatrix}
%     \vec{y}₁^\prime \\
%     \vec{y}₂^\prime \\
%     … \\
%     \vec{y}_{n - 1}^\prime \\
%     \vec{y}_n^\prime
%   \end{pmatrix}
%   =
%   \begin{pmatrix}
%     \vec{y}₂\\
%     \vec{y}₃\\
%     … \\
%     \vec{y}ₙ\\
%     f(x, \vec{y}₁, \vec{y}₂, …, \vec{y}ₙ)
%   \end{pmatrix}.
% \end{equation}
% Damit ist es ausreichend, nur Differentialgleichungen 1. Ordnung von der Form:
% \begin{equation}
%   \label{eq:3}
%   \vec{y}^\prime = f(x, \vec{y}),
% \end{equation}
% numerisch zu lösen.

\begin{question}[subtitle=Euler-Verfahren]
  In der Physik (vorallem in der Mechanik) wird durch eine Bewegungsgleichungen in differentieller Form
  \begin{equation}
    \label{eq:5}
    \dot{\vec{r}} = f(t, \vec{r}),
  \end{equation}
  der zeitlichen Verlauf einer Ortskoordinate $\vec{r}(t)$ mit der Anfangsbedingung $\vec{y}(\vec{0}) = \vec{y}₀$ eindeutig festgelegt.
  Um diese Differentialgleichung 1. Ordnung numerisch zu lösen, wird das Zeitintervall der Lösung $[0, T]$ in $N$-Schritten durch eine \textit{feste} Schrittbreite $h = \sfrac{T}{N}$ unterteilt.
  Gesucht sind dann bei der diskreten Zeit $tₙ = n h$ die Funktionswerte $\vec{r}ₙ = \vec{r}(tₙ)$.

  Das einfachste Lösungsverfahren ist durch das sogenannte Euler-Verfahren gegeben, das auf einer einfachen Diskretisierung der Zeitableitung auf der linken Seite der Differentialgleichung in~\eqref{eq:5} beruht,
  \begin{equation}
    \label{eq:6}
    \vec{y}_{n+1} = \vec{y}_n + h \vec{f}\left( tₙ, \vec{y}ₙ \right) + \symcal{O}(h²).
  \end{equation}

  \begin{enumerate}[(i)]
  \item Schreiben Sie ein Programm, dass die Newtonsche Bewegungsgleichung für $\vec{r}(t)$ und $\vec{v}(t)$ eines Teilchen mit der Masse $m$ in einem Kraftfeld $\vec{F}(\vec{r})$,
    \begin{align}
      \dot{\vec{r}} & = \vec{v}, \\
      \dot{\vec{v}} & = \frac{1}{m} \vec{F} (\vec{r}),
    \end{align}
    mit Hilfe des Euler-Verfahren mit fester Schrittbreite löst.
    Das Programm sollte modular gestaltet sein, sodass Ihr Bewegungsgleichungslöser wiederverwendet werden kann.
    Darüberhinaus sollte in geeigneter Weise das Kraftfeld $\vec{F}(\vec{r})$ als Argument an die Löser-Prozedur übergeben werden können und Ihr Löser sollte dabei auf mindestens drei Raumdimensionen ($\vec{r}, \vec{v}, \vec{F} ∈ \symbb{R}³$) oder einer beliebigen Dimension $d$ arbeiten können.

    \textit{Hinweis:} Beachten Sie, dass Sie die beiden Differentialgleichung zur gleichen Zeit $tₙ$ lösen.
  \item Testen Sie Ihren Bewegungsgleichungslöser anhand des harmonische Oszillators,
    \begin{equation}
      \label{eq:7}
      \frac{1}{m} \vec{F}(\vec{r}) = - \vec{r}.
    \end{equation}
    \begin{enumerate}[a)]
    \item Testen Sie für die Anfangsbedingungen: $\vec{r}(0)= (-1, +1)^{T}$, $\vec{v}(0) = (0, 0)^{T}$, dass Sie eine harmonische Schwingung erhalten.

      \textit{Hinweis:} Geben Sie hierzu $tₙ, \vec{r}ₙ$ und $\vec{v}ₙ$ auf der Standardausgabe aus. Ein ausführbares Skript zum Plotten der Datenpunkte wird Ihnen zur Verfügung gestellt.
    \item Testen Sie, wie sich verschiedene Schrittweiten auf die maximale Auslenkung des Oszillator über mehrere Perioden auswirkt.
      Was beobachten Sie und welche physikalische Auswirkung könnte dies haben?
    \item Testen Sie die Energieerhaltung, d.\,h.\ $H = \frac{1}{2}m\dot{\vec{r}}² + V(\vec{r}) = \text{const}$.\\
      \textit{Hinweis:} Es gilt $\vec{F}(\vec{r}) = - ∇V(\vec{r}) ⇒ V(\vec{r}) = \frac{1}{2}r²$.
    \end{enumerate}
  \end{enumerate}
\end{question}

\textit{Hinweis zur Abgabe:}
\begin{itemize}
\item Gliedern Sie Ihre Lösung sinnvoll und benutzen Sie verständliche Modul-, Prozedur-, Typ- oder Variablennamen und versehen Sie entsprechende Stellen mit aussagreichen Kommentaren.
\item Alle eingereichten Quellcode-Dateien sollte mit folgenden Compiler-Flags unter \verb|gfortran| fehlerfrei kompilieren: \verb|-O2 -Wall -Werror -pedantic -std=f2008ts|.
\item Sie müssen kein Makefile o.\,ä.\ bereitstellen.
\item Senden Sie Ihre Lösung an die E-Mail: \url{brass@physik.uni-siegen.de}.
\end{itemize}
\end{document}
