\documentclass{scrartcl}

% Paket float verbessern
\usepackage{scrhack}

% Kleine Anpassung von maketitle
\usepackage{titling}
% \usepackage{showframe}

% Warnung, falls nochmal kompiliert werden muss
\usepackage[aux]{rerunfilecheck}

% unverzichtbare Mathe-Befehle
\usepackage{amsmath}
% viele Mathe-Symbole
\usepackage{amssymb}
% Erweiterungen für amsmath
\usepackage{mathtools}

% Fonteinstellungen
\usepackage{selnolig}
\usepackage{fontspec}
% Latin Modern Fonts werden automatisch geladen
% Alternativ zum Beispiel:
%\setromanfont{Libertinus Serif}
%\setsansfont{Libertinus Sans}
%\setmonofont{Libertinus Mono}

% Wenn man andere Schriftarten gesetzt hat,
% sollte man das Seiten-Layout neu berechnen lassen
\recalctypearea{}

% deutsche Spracheinstellungen
\usepackage{polyglossia}
\setmainlanguage{german}
\setotherlanguage{english}

\usepackage[
  math-style=ISO,    % ┐
  bold-style=ISO,    % │
  sans-style=italic, % │ ISO-Standard folgen
  nabla=upright,     % │
  partial=upright,   % ┘
  warnings-off={           % ┐
    mathtools-colon,       % │ unnötige Warnungen ausschalten
    mathtools-overbracket, % │
  },                       % ┘
]{unicode-math}

% traditionelle Fonts für Mathematik
\setmathfont{Latin Modern Math}
% Alternativ zum Beispiel:
%\setmathfont{Libertinus Math}

\setmathfont{XITS Math}[range={scr, bfscr}]
\setmathfont{XITS Math}[range={cal, bfcal}, StylisticSet=1]

% Zahlen und Einheiten
\usepackage[
  locale=DE,                   % deutsche Einstellungen
  separate-uncertainty=true,   % immer Fehler mit \pm
  per-mode=symbol-or-fraction, % / in inline math, fraction in display math
]{siunitx}

% chemische Formeln
\usepackage[
  version=4,
  math-greek=default, % ┐ mit unicode-math zusammenarbeiten
  text-greek=default, % ┘
]{mhchem}

% richtige Anführungszeichen
\usepackage[autostyle]{csquotes}

% schöne Brüche im Text
\usepackage{xfrac}

% Standardplatzierung für Floats einstellen
\usepackage{float}
\floatplacement{figure}{htbp}
\floatplacement{table}{htbp}

% Floats innerhalb einer Section halten
\usepackage[
  section, % Floats innerhalb der Section halten
  below,   % unterhalb der Section aber auf der selben Seite ist ok
]{placeins}

% Seite drehen für breite Tabellen: landscape Umgebung
\usepackage{pdflscape}

% Captions schöner machen.
\usepackage[
  labelfont=bf,        % Tabelle x: Abbildung y: ist jetzt fett
  font=small,          % Schrift etwas kleiner als Dokument
  width=0.9\textwidth, % maximale Breite einer Caption schmaler
]{caption}
% subfigure, subtable, subref
\usepackage{subcaption}

% Grafiken können eingebunden werden
\usepackage{graphicx}
% größere Variation von Dateinamen möglich
\usepackage{grffile}

% schöne Tabellen
\usepackage{booktabs}

% Verbesserungen am Schriftbild
\usepackage{microtype}

% Literaturverzeichnis
\usepackage[
  backend=biber,
]{biblatex}
% Quellendatenbank
% \addbibresource{lit.bib}
% \addbibresource{programme.bib}

% Hyperlinks im Dokument
\usepackage[
  unicode,        % Unicode in PDF-Attributen erlauben
  pdfusetitle,    % Titel, Autoren und Datum als PDF-Attribute
  pdfcreator={},  % ┐ PDF-Attribute säubern
  pdfproducer={}, % ┘
]{hyperref}
% erweiterte Bookmarks im PDF
\usepackage{bookmark}
\usepackage{cleveref} % muss nach hyperref kommen

% Trennung von Wörtern mit Strichen
\usepackage[shortcuts]{extdash}

\usepackage{expl3}
\usepackage{xparse}

\usepackage{minted}
\usepackage{enumerate}
\usepackage{exsheets}
\SetupExSheets{
  headings=block-subtitle,
  question/name=Aufgabe,
}

\DeclareInstance{exsheets-heading}{block-subtitle}{default}{
  join = {
    title[r,B]number[l,B](.333em,0pt) ;
    title[r,B]subtitle[l,B](1em,0pt)
  } ,
  attach = {
    main[l,vc]title[l,vc](0pt,0pt) ;
    main[r,vc]points[l,vc](\marginparsep,0pt)
  }
}

\setlength{\droptitle}{-4em}
\pretitle{\begin{center}\Large}
\posttitle{\par\end{center}}
\preauthor{
  \begin{center}\large \lineskip 0.2em
    \begin{tabular}[t]{c}
}
\postauthor{\end{tabular}\par\end{center}}
\predate{\begin{center}\normalsize}
\postdate{\par\end{center}}

\setlength{\parindent}{0cm}
\setlength{\topmargin}{0in}
\setlength{\headheight}{0cm}
\setlength{\headsep}{0cm}
\setlength{\textheight}{9.0in}
\setlength{\evensidemargin}{0.0in}
\setlength{\oddsidemargin}{0.0in}
\setlength{\textwidth}{6in}
\setlength{\footskip}{.8in}

\title{\textsf{\textbf{Übungen zum wissenschaftlichen Programmieren SS2019}}}
\author{Prof.\,Dr.\,W.\,Kilian\unskip, S.\,Braß}

\ExplSyntaxOn
\RenewDocumentCommand \maketitlehooka {} {
  \rule{\textwidth}{1pt}
}
\RenewDocumentCommand \maketitlehookd {} {
  \rule{\textwidth}{1pt}
}
\RenewDocumentCommand \maketitlehookc {} {
  \begin{center}
    \large \AufgabenBlatt
  \end{center}
}
\NewDocumentCommand \I {}
{
\symup{i}
}
\NewDocumentCommand \Sp {}
{
  \operatorname{Sp}
}
\NewDocumentCommand \adj {}
{
  \operatorname{adj}
}
\ExplSyntaxOff

\NewDocumentCommand\AufgabenBlatt{}{Übungsblatt 3}
\date{Ausgabe: Di, 30.04.2019, Besprechung: Fr, 03.05.2019}
\setcounter{question}{5}
\begin{document}

\maketitle

\begin{question}[subtitle=Erzeugung von beliebigen Verteilungen]
  Ein Zufallszahlengenerator, der gleichverteilte Zufallszahlen auf dem Intervall $[0, 1]$ erzeugt, kann auch eingesetzt werden, um beliebige Verteilungen zu erzeugen.

  \begin{description}
  \item[Die Inversionsmethode] definiert die Quantilenfunktion
    \begin{equation}
      F^{-1} ≔ \inf \{ x ∈ Ω | F(x) ≥ u \}
    \end{equation}
    für eine Wahrscheinlichkeitsverteilung $F(x)$ und der dazugehörigen Wahrscheinlichkeitsdichte $f(x) = \sfrac{\symup{d}}{\symup{dx}} F(x)$.
    Sei nun $u ∈ [0, 1]$ eine gleichverteilte Zufallszahl, dann ist $x = F^{-1}(u)$ eine reelle Zufallszahl, die der Verteilungsfunktion $F(x)$ genügt.
    Weiterhin, sei die Verteilungsfunktions $F: Ω \mapsto [0, 1]$ \textit{streng} monoton wachsend, dann stimmt die Quantilenfunktion $F^{-1}(x)$ mit der analytischen Umkehrfunktion $F^{-1}: [0, 1] \mapsto Ω$ überein.
  \item[Der Box--Muller-Algorithmus] wird angewandt, um eine Gauß-Verteilung zu erzeugen, für welche es keine analytisch-geschlossene Umkehrfunktion gibt.
    Mit zwei, gleichverteilten Zufallszahlen $u₁, u₂ ∈ [0, 1]$ können durch
    \begin{align}
      y₁ & = \sqrt{-2 \log u₁} \cos(2π u₂), \\
      y₂ & = \sqrt{-2 \log u₁} \sin(2π u₂),
    \end{align}
    zwei unabbhängige, Gauß-verteile Zufallszahlen $y₁, y₂$ mit Mittelwert $μ = 0$ und Standardabweichung $σ² = 1$ erzeugt werden.

  \item[Der zentrale Grenzwert-Satz] kann genutzt werden, um Gauß-verteilte Zufallszahlen zu erzeugen:
    \blockquote{Die Summe vieler, unabhängiger, identisch verteilter Zufallsvariablen ist Gauß-verteilt.}
    Um Gauß-verteilte Zufallszahlen $y$ zu erhalten, werden $N$ gleichverteilte Zufallsvariablen $uᵢ ∈ [0, 1]$ zu $y = ∑_{i = 1}^N uᵢ - \sfrac{N}{2}$ addiert, mit Mittelwert $μ = 0$ und $σ² = \sfrac{N}{12}$.

  \item[Die Rückweisungsmethode] kann für beliebige Verteilungsfunktionen angewendet werden, insbesondere, wo die Umkehrfunktion nicht bekannt ist oder es sonst keine Möglichkeit zur Bestimmung der Verteilung gibt.
    Die zu erzeugende Verteilung mit Wahrscheinlichkeitsdichte $f(x)$ wird durch eine Vergleichswahrscheinlichkeitsdichte $g(x)$ und der Konstante $C > 1$ eingehüllt, sodass gilt, $p(x) ≤ C · g(x), ∀x$.
    Die Vergleichswahrscheinlichkeitsdichte $g(x)$ sollte dabei sogewählt werden, dass man Zufallszahlen $x$ entsprechend von $g(x)$ mit der Transformationsmethode erzeugen kann.

    Der Algorithmus für die Rückweisungsmethoden lautet:
    \begin{enumerate}[(i)]
    \item Ziehe eine Zufallszahl $x$ aus der Verteilung $g(x)$.
    \item Ziehe eine Zufallszahl $y$, die gleichverteilt im Interval $[0, C g(x)]$ ist.
    \item Akzeptiere das Tupel $(x, y)$, wenn $y ≤ p(x)$, dann ist $x$ gemäß $p(x)$ verteilt. Ansonsten, verwerfe das Tupel und wiederhole.
    \end{enumerate}
    Desto ähnlicher sich $f(x)$ und $g(x)$ sind desto höher ist die Akzeptanzrate (Effizienz) des Algorithmus.
  \end{description}

  Erzeugen Sie folgende Verteilung:
  \begin{enumerate}[(i)]
  \item Benutzen Sie den Box--Muller-Algorithmus, um eine Gauß-Verteilung mit Varianz \num{1} und Mittelwert \num{0} zu erzeugen.
  \item Benutzen Sie den zentralen Grenzwert-Satz, um eine Gauß-Verteilung zu erzeugen.
    Bilden Sie hierfür die Summe von $N$ (geeignet wählen) gleichverteilten Zufallszahlen aus dem Intervall $[0, 1]$. Wie bekommt man eine Verteilung mit Mittelwert \num{0} und Standardabweichung \num{0}?
    Welche Nachteile hat diese Methode, z.\, B.\ in Korrekteit und Effizienz?
  \item Benutzen Sie die Rückweisungsmethode, um die Verteilung $p₁(x) = \sfrac{\sin(x)}{2}$ in den Grenzen \num{0}~bis~$π$ zu erzeugen.
  \item Benutzen Sie die Transformationsmethode, um die Verteilung $p₂(x) = 3x²$ in den Grenzen \num{0}~bis~\num{1} zu erzeugen.
  \end{enumerate}
  Erzeugen Sie jeweils $10⁴$ Zufallszahlen und erstellen Sie ein Histogram, wo die Häufigkeit $p(xᵢ)$ der um $xᵢ$ zentrierten Bins der Länge $Δx$ gemäß $∑ᵢ p(xᵢ)Δx$ normiert werden sollen.
  Plotten Sie auch die zugehörige normierte analytische Verteilung.
\end{question}

\begin{question}[subtitle=$χ²$- oder Gleichverteilungstest]
  Pseudo-Zufallszahlengeneratoren sollten neben dem Korrelationstest von Marsaglia mit einer Vielzahl weiterer statistischer Tests untersucht werden.

  Ein weiterer solcher Test ist der $χ²$- oder Gleichverteilungstest.
  Hierfür wir das Intervall $[0, 1]$ in $M$ Abschnitte gleicher Länge unterteilt.
  Es werden $N$ Zufallszahlen generiert und über die $M$ Abschnitte histogrammiert, so dass für jeden Abschnitt $i ∈ \{1, …, M\}$ die Anzahl der Zufallszahlen $nᵢ$ bekannt ist.

  Der erwartete Mittelwert über viele Iterationen für $nᵢ$ ist
  \begin{equation}
    \langle nᵢ \rangle = \frac{N}{M},
  \end{equation}
  da die $nᵢ$ Poisson-verteilt sein sollen, ist die Varianz $V(nᵢ) = \langle (nᵢ - \langle nᵢ \rangle)² \rangle = \langle nᵢ \rangle = \frac{N}{M}$.

  Aus der Nebenbedingung, dass für die $M$ Werte $nᵢ$ gilt $∑_{i = 1}^M nᵢ = N$, folgt, dass es nur $(M - 1)$ Freiheitsgrade gibt.
  Damit lässt sich der $χ²$-Wert bestimmen
  \begin{equation}
    χ² = ∑_{i = 1}^M \frac{(nᵢ - \langle nᵢ \rangle)²}{\langle  nᵢ \rangle} = M - 1.
  \end{equation}
  Eine Abweichung von diesem Wert, deutet daraufhin, dass die Zufallszahlen nicht gleichverteilt sind.

  \begin{enumerate}[(i)]
  \item Implementieren Sie das Histogramm über die $nᵢ$ als ein-dimensionalen Integer-Array mit der Größe $M$.
    Bestimmen Sie den entsprechend Abschnitt $k$ mit $k = \lfloor r \rfloor + 1$ und benutzen Sie als Gewicht \num{1} zum Histogrammieren.
  \item Implementieren Sie den Mittelwert und die Varianz von $nᵢ$ mithilfe von einem zwei-dimensionalen Array der Größe $(2, M)$, wobei in den Indizes $(1, i)$ die Mittelwerte von $nᵢ$ und in den Indizes $(2, i)$ die Varianz von $nᵢ$ gespeichert werden sollen. Nutzen Sie Welford's Algorithmus zur Bestimmung des Mittelwertes und der Varianz von $nᵢ$.
    Bestimmen Sie die Mittelwerte und die Varianzen für $nᵢ$ und sowie $χ²$ über \num{15} Iterationen mit den Werten $M = 100$ und $N = 1000$ für jeweils einen (ausgewählten) linear kongruenten Generator und einen \textsc{Xorshift}-Generator aus Übungsblatt~2.
  \item Entsprechen bei beiden Zufallszahlengeneratoren die Mittelwerte und die Varianzen von $nᵢ$ mit der Vorhersage überein? Bestehen beide den $χ²$-Test?
  \end{enumerate}
  \inputminted{fortran}{../src/aufgabe07.f90}
\end{question}
\end{document}