\documentclass{scrartcl}

% Paket float verbessern
\usepackage{scrhack}

% Kleine Anpassung von maketitle
\usepackage{titling}
% \usepackage{showframe}

% Warnung, falls nochmal kompiliert werden muss
\usepackage[aux]{rerunfilecheck}

% unverzichtbare Mathe-Befehle
\usepackage{amsmath}
% viele Mathe-Symbole
\usepackage{amssymb}
% Erweiterungen für amsmath
\usepackage{mathtools}

% Fonteinstellungen
\usepackage{selnolig}
\usepackage{fontspec}
% Latin Modern Fonts werden automatisch geladen
% Alternativ zum Beispiel:
%\setromanfont{Libertinus Serif}
%\setsansfont{Libertinus Sans}
%\setmonofont{Libertinus Mono}

% Wenn man andere Schriftarten gesetzt hat,
% sollte man das Seiten-Layout neu berechnen lassen
\recalctypearea{}

% deutsche Spracheinstellungen
\usepackage{polyglossia}
\setmainlanguage{german}
\setotherlanguage{english}

\usepackage[
  math-style=ISO,    % ┐
  bold-style=ISO,    % │
  sans-style=italic, % │ ISO-Standard folgen
  nabla=upright,     % │
  partial=upright,   % ┘
  warnings-off={           % ┐
    mathtools-colon,       % │ unnötige Warnungen ausschalten
    mathtools-overbracket, % │
  },                       % ┘
]{unicode-math}

% traditionelle Fonts für Mathematik
\setmathfont{Latin Modern Math}
% Alternativ zum Beispiel:
%\setmathfont{Libertinus Math}

\setmathfont{XITS Math}[range={scr, bfscr}]
\setmathfont{XITS Math}[range={cal, bfcal}, StylisticSet=1]

% Zahlen und Einheiten
\usepackage[
  locale=DE,                   % deutsche Einstellungen
  separate-uncertainty=true,   % immer Fehler mit \pm
  per-mode=symbol-or-fraction, % / in inline math, fraction in display math
]{siunitx}

% chemische Formeln
\usepackage[
  version=4,
  math-greek=default, % ┐ mit unicode-math zusammenarbeiten
  text-greek=default, % ┘
]{mhchem}

% richtige Anführungszeichen
\usepackage[autostyle]{csquotes}

% schöne Brüche im Text
\usepackage{xfrac}

% Standardplatzierung für Floats einstellen
\usepackage{float}
\floatplacement{figure}{htbp}
\floatplacement{table}{htbp}

% Floats innerhalb einer Section halten
\usepackage[
  section, % Floats innerhalb der Section halten
  below,   % unterhalb der Section aber auf der selben Seite ist ok
]{placeins}

% Seite drehen für breite Tabellen: landscape Umgebung
\usepackage{pdflscape}

% Captions schöner machen.
\usepackage[
  labelfont=bf,        % Tabelle x: Abbildung y: ist jetzt fett
  font=small,          % Schrift etwas kleiner als Dokument
  width=0.9\textwidth, % maximale Breite einer Caption schmaler
]{caption}
% subfigure, subtable, subref
\usepackage{subcaption}

% Grafiken können eingebunden werden
\usepackage{graphicx}
% größere Variation von Dateinamen möglich
\usepackage{grffile}

% schöne Tabellen
\usepackage{booktabs}

% Verbesserungen am Schriftbild
\usepackage{microtype}

% Literaturverzeichnis
\usepackage[
  backend=biber,
]{biblatex}
% Quellendatenbank
% \addbibresource{lit.bib}
% \addbibresource{programme.bib}

% Hyperlinks im Dokument
\usepackage[
  unicode,        % Unicode in PDF-Attributen erlauben
  pdfusetitle,    % Titel, Autoren und Datum als PDF-Attribute
  pdfcreator={},  % ┐ PDF-Attribute säubern
  pdfproducer={}, % ┘
]{hyperref}
% erweiterte Bookmarks im PDF
\usepackage{bookmark}
\usepackage{cleveref} % muss nach hyperref kommen

% Trennung von Wörtern mit Strichen
\usepackage[shortcuts]{extdash}

\usepackage{expl3}
\usepackage{xparse}

\usepackage{minted}
\usepackage{enumerate}
\usepackage{exsheets}
\SetupExSheets{
  headings=block-subtitle,
  question/name=Aufgabe,
}

\DeclareInstance{exsheets-heading}{block-subtitle}{default}{
  join = {
    title[r,B]number[l,B](.333em,0pt) ;
    title[r,B]subtitle[l,B](1em,0pt)
  } ,
  attach = {
    main[l,vc]title[l,vc](0pt,0pt) ;
    main[r,vc]points[l,vc](\marginparsep,0pt)
  }
}

\setlength{\droptitle}{-4em}
\pretitle{\begin{center}\Large}
\posttitle{\par\end{center}}
\preauthor{
  \begin{center}\large \lineskip 0.2em
    \begin{tabular}[t]{c}
}
\postauthor{\end{tabular}\par\end{center}}
\predate{\begin{center}\normalsize}
\postdate{\par\end{center}}

\setlength{\parindent}{0cm}
\setlength{\topmargin}{0in}
\setlength{\headheight}{0cm}
\setlength{\headsep}{0cm}
\setlength{\textheight}{9.0in}
\setlength{\evensidemargin}{0.0in}
\setlength{\oddsidemargin}{0.0in}
\setlength{\textwidth}{6in}
\setlength{\footskip}{.8in}

\title{\textsf{\textbf{Übungen zum wissenschaftlichen Programmieren SS2019}}}
\author{Prof.\,Dr.\,W.\,Kilian\unskip, S.\,Braß}

\ExplSyntaxOn
\RenewDocumentCommand \maketitlehooka {} {
  \rule{\textwidth}{1pt}
}
\RenewDocumentCommand \maketitlehookd {} {
  \rule{\textwidth}{1pt}
}
\RenewDocumentCommand \maketitlehookc {} {
  \begin{center}
    \large \AufgabenBlatt
  \end{center}
}
\NewDocumentCommand \I {}
{
\symup{i}
}
\NewDocumentCommand \Sp {}
{
  \operatorname{Sp}
}
\NewDocumentCommand \adj {}
{
  \operatorname{adj}
}
\ExplSyntaxOff

\NewDocumentCommand\AufgabenBlatt{}{Übungsblatt 4}
\date{Ausgabe: Fr, 03.05.2019, Besprechung: Fr, 10.05.2019}
\setcounter{question}{6}
\begin{document}

\maketitle

\begin{question}[subtitle=Fixpunktsatz von Banach]
  Der Fixpunktsatz von Banach gibt an unter welchen Bedingungen eine selbst-abbildende Funktionsvorschrift gegen einen Fixpunkt konvergiert.
  Eine Anwendung des Fixpunktsatz von Banach sind die Methoden zur Bestimmung von Nullstellen, bzw. das Lösen von nicht-linearen Gleichungen.

  Ein solches Lösungsverfahren stellt die Methode der Intervallhalbierung dar:
  \begin{enumerate}[1)]
  \item Angenommen, es seien Nullstelle(n) in einem Intervall \textquote{eingeklammert}, d.h. $x₀$ und $y₀ (> x₀)$ liegen so, dass $f(x)$ auf $[x₀, y₀]$ einen Vorzeichenwechsel aufweist: $f(x₀) · f(y₀) < 0$.
  \item Berechne $z_{n + 1} = \frac{x_n + y_n}{2}$.
  \item Wenn $f(z_{n + 1}) · f(x_n) < 0$, dann setze $x_{n + 1} = x_n ∧ y_{n + 1} = z_{n + 1}$, sonst setze $x_{n + 1} = z_{n + 1} ∧ y_{n + 1} = y_n$.
  \item Breche ab, wenn das Genauigkeitsziel $ε_{n + 1} ≔ y_{n + 1} - x_{n + 1} < ε$ erreicht ist, ansonsten wiederhole ab 2).
  \end{enumerate}

  Finden Sie alle reellen Lösungen zu folgenden Gleichungen (wählen Sie als Genauigkeitsziel $ε = 10^{-8}$):
  \begin{enumerate}[(i)]
  \item $f(x) = 0.5 x³ - 2 x² + x + 1 = 0$,
  \item $g(x) = h(x)$, mit $g(x) = 3x + \sin x ∧ h(x) = \exp x$.
  \end{enumerate}
\end{question}

\begin{question}[subtitle=Bifurkationsdiagramme]
  Iterative Verfahren sollen meist dazuführen, dass sie zu einer eindeutigen Lösung, dem Fixpunkt, konvergieren.
  Die Voraussetzung für die Konvergenz liefert der Fixpunktsatz von Banach.
  Was passiert, wenn eine Abildung diese Voraussetzung nicht auf ihrem ganzen Definitionsbereich erfüllt, kann anhand der logistischen und der kubischen Abbildung untersucht werden.

  Berechnen und plotten Sie die Bifurkationsdiagramme, wo Sie $x_n$ gegen $r$ auftragen, für die Abbildungen
  \begin{align}
    x_{n + 1} & = r x_n (1 - x_n), \, x ∈ [0, 1],\, \text{(logistische Abbildung)}, \\
    x_{n + 1} & = r x_n - x_n³, \, x_n ∈ [-\sqrt{1 + r}, \sqrt{1 + r}], \, \text{(kubische Abbildung)},
  \end{align}
  durch numerische Iteration.

  Gehen Sie dabei folgendermaßen vor:
  \begin{enumerate}[(i)]
  \item  Vergrößern Sie den Parameter $r$ in kleinen Schritten in einem Bereich $0 < r < r_{\text{max}}$.
    Was passiert, wenn Sei $r$ zu groß wählen und welche $r_{\text{max}}$ ergeben sich?
  \item Wählen Sie einen Startpunkt $x₀$ im Intervall $[0, 1]$ zufällig aus (Fortran: \mintinline{fortran}|random_number|) und iterieren Sie für jedes $r$ die Abbildung für $N = 100$, damit sich ein Fixpunkt oder ein sogenannter Orbit einstellt.
    Iterieren Sie dann für weitere $M = 300$, jeder Punkt eines Orbits ergibt einen Punkt im Bifurkationsdiagramm.
  \end{enumerate}
\end{question}

\begin{question}[subtitle=Feigenbaum-Konstante]
  Die Fixpunkt-Gleichung für die $2^n$-fach iterierte logistische Abbildung $f(r, x) = rx(1 - x)$,
  \begin{equation}
    x_{2^n} = f^{2^n}(r, x) = f(r, f(r, …f(r, x₀))),
  \end{equation}
  gibt für $r < r_∞ = 3.569945…$ die Werte eines Orbits der Länge $2^n$ im Periodenverdopplungsszenario nach Feigenbaum.

  Bestimmen Sie für $n = 1, 2, 3$ numerisch die Werte $r = R_n < r_∞$, für die superstabile Fixpunkte existieren. Diese Werte erfüllen die Gleichung
  \begin{equation}
    \frac{1}{2} = f^{2^n}\left( r, \frac{1}{2} \right).
  \end{equation}
  \begin{enumerate}[(i)]
  \item Plotten Sie zunächst $g_n(r)≡\sfrac{1}{2} - f^{2^n} (r, \sfrac{1}{2})$ als Funktion von $r$ für $n = 0, 1, 2, 3$ im Bereich $0 < r < r_∞$.
    Wird $n$ um \num{1} vergrößert, kommt jeweils eine Nullstelle von $g_n(r)$ bei $r = R_n$ hinzu.
    Machen Sie die Schrittweite in ihrem Plot so klein, dass Sie Schranken für die Nullstellen angeben können.
  \item Bestimmen Sie numerisch mit Hilfe der Methode der Intervallhalbierung ausgehend von diesen Schranken die Nullstellen $R_n$ von $g_n(r)$ für $n = 0, 1, 2, 3$.
  \item Gewinnen Sie aus Ihren Ergebnissen eine erste Schätzung der Feigenbaum-Konstante:
    \begin{equation}
      δ = \lim_{n → ∞} \frac{R_{n - 1} - R_{n - 2}}{R_n - R_{n - 1}}.
    \end{equation}
  \end{enumerate}
\end{question}
\end{document}