\documentclass{scrartcl}

% Paket float verbessern
\usepackage{scrhack}

% Kleine Anpassung von maketitle
\usepackage{titling}
% \usepackage{showframe}

% Warnung, falls nochmal kompiliert werden muss
\usepackage[aux]{rerunfilecheck}

% unverzichtbare Mathe-Befehle
\usepackage{amsmath}
% viele Mathe-Symbole
\usepackage{amssymb}
% Erweiterungen für amsmath
\usepackage{mathtools}

% Fonteinstellungen
\usepackage{selnolig}
\usepackage{fontspec}
% Latin Modern Fonts werden automatisch geladen
% Alternativ zum Beispiel:
%\setromanfont{Libertinus Serif}
%\setsansfont{Libertinus Sans}
%\setmonofont{Libertinus Mono}

% Wenn man andere Schriftarten gesetzt hat,
% sollte man das Seiten-Layout neu berechnen lassen
\recalctypearea{}

% deutsche Spracheinstellungen
\usepackage{polyglossia}
\setmainlanguage{german}
\setotherlanguage{english}

\usepackage[
  math-style=ISO,    % ┐
  bold-style=ISO,    % │
  sans-style=italic, % │ ISO-Standard folgen
  nabla=upright,     % │
  partial=upright,   % ┘
  warnings-off={           % ┐
    mathtools-colon,       % │ unnötige Warnungen ausschalten
    mathtools-overbracket, % │
  },                       % ┘
]{unicode-math}

% traditionelle Fonts für Mathematik
\setmathfont{Latin Modern Math}
% Alternativ zum Beispiel:
%\setmathfont{Libertinus Math}

\setmathfont{XITS Math}[range={scr, bfscr}]
\setmathfont{XITS Math}[range={cal, bfcal}, StylisticSet=1]

% Zahlen und Einheiten
\usepackage[
  locale=DE,                   % deutsche Einstellungen
  separate-uncertainty=true,   % immer Fehler mit \pm
  per-mode=symbol-or-fraction, % / in inline math, fraction in display math
]{siunitx}

% chemische Formeln
\usepackage[
  version=4,
  math-greek=default, % ┐ mit unicode-math zusammenarbeiten
  text-greek=default, % ┘
]{mhchem}

% richtige Anführungszeichen
\usepackage[autostyle]{csquotes}

% schöne Brüche im Text
\usepackage{xfrac}

% Standardplatzierung für Floats einstellen
\usepackage{float}
\floatplacement{figure}{htbp}
\floatplacement{table}{htbp}

% Floats innerhalb einer Section halten
\usepackage[
  section, % Floats innerhalb der Section halten
  below,   % unterhalb der Section aber auf der selben Seite ist ok
]{placeins}

% Seite drehen für breite Tabellen: landscape Umgebung
\usepackage{pdflscape}

% Captions schöner machen.
\usepackage[
  labelfont=bf,        % Tabelle x: Abbildung y: ist jetzt fett
  font=small,          % Schrift etwas kleiner als Dokument
  width=0.9\textwidth, % maximale Breite einer Caption schmaler
]{caption}
% subfigure, subtable, subref
\usepackage{subcaption}

% Grafiken können eingebunden werden
\usepackage{graphicx}
% größere Variation von Dateinamen möglich
\usepackage{grffile}

% schöne Tabellen
\usepackage{booktabs}

% Verbesserungen am Schriftbild
\usepackage{microtype}

% Literaturverzeichnis
\usepackage[
  backend=biber,
]{biblatex}
% Quellendatenbank
% \addbibresource{lit.bib}
% \addbibresource{programme.bib}

% Hyperlinks im Dokument
\usepackage[
  unicode,        % Unicode in PDF-Attributen erlauben
  pdfusetitle,    % Titel, Autoren und Datum als PDF-Attribute
  pdfcreator={},  % ┐ PDF-Attribute säubern
  pdfproducer={}, % ┘
]{hyperref}
% erweiterte Bookmarks im PDF
\usepackage{bookmark}
\usepackage{cleveref} % muss nach hyperref kommen

% Trennung von Wörtern mit Strichen
\usepackage[shortcuts]{extdash}

\usepackage{expl3}
\usepackage{xparse}

\usepackage{minted}
\usepackage{enumerate}
\usepackage{exsheets}
\SetupExSheets{
  headings=block-subtitle,
  question/name=Aufgabe,
}

\DeclareInstance{exsheets-heading}{block-subtitle}{default}{
  join = {
    title[r,B]number[l,B](.333em,0pt) ;
    title[r,B]subtitle[l,B](1em,0pt)
  } ,
  attach = {
    main[l,vc]title[l,vc](0pt,0pt) ;
    main[r,vc]points[l,vc](\marginparsep,0pt)
  }
}

\setlength{\droptitle}{-4em}
\pretitle{\begin{center}\Large}
\posttitle{\par\end{center}}
\preauthor{
  \begin{center}\large \lineskip 0.2em
    \begin{tabular}[t]{c}
}
\postauthor{\end{tabular}\par\end{center}}
\predate{\begin{center}\normalsize}
\postdate{\par\end{center}}

\setlength{\parindent}{0cm}
\setlength{\topmargin}{0in}
\setlength{\headheight}{0cm}
\setlength{\headsep}{0cm}
\setlength{\textheight}{9.0in}
\setlength{\evensidemargin}{0.0in}
\setlength{\oddsidemargin}{0.0in}
\setlength{\textwidth}{6in}
\setlength{\footskip}{.8in}

\title{\textsf{\textbf{Übungen zum wissenschaftlichen Programmieren SS2019}}}
\author{Prof.\,Dr.\,W.\,Kilian\unskip, S.\,Braß}

\ExplSyntaxOn
\RenewDocumentCommand \maketitlehooka {} {
  \rule{\textwidth}{1pt}
}
\RenewDocumentCommand \maketitlehookd {} {
  \rule{\textwidth}{1pt}
}
\RenewDocumentCommand \maketitlehookc {} {
  \begin{center}
    \large \AufgabenBlatt
  \end{center}
}
\NewDocumentCommand \I {}
{
\symup{i}
}
\NewDocumentCommand \Sp {}
{
  \operatorname{Sp}
}
\NewDocumentCommand \adj {}
{
  \operatorname{adj}
}
\ExplSyntaxOff

\NewDocumentCommand\AufgabenBlatt{}{Übungsblatt 5}
\date{Ausgabe: Fr, 10.05.2019, Besprechung: Fr, 17.05.2019}
\setcounter{question}{9}
\begin{document}

\maketitle

\begin{question}[subtitle=Numerische Differentiation]
Die erste Ableitung einer Funktion $f(x)$ wird durch den kontinuierlichen Grenzprozess
\begin{equation}
  f^{\prime}(x) = \lim_{h → 0} \frac{f(x+h) - f(x)}{h}
\end{equation}
definiert, welcher im Computer nicht ausgeführt werden kann.
Für numerische Anwendungen kann der Differenzenquotient daher naiv
\begin{equation}
  f^{\prime}(x) = \frac{f(x + h) - f(x)}{h} + \symcal{O}(h)
\end{equation}
mit einem hinreichend kleinem $h$ angenähert werden.
Der Abbruchfehler $R(f^{\prime})$ ist durch
\begin{equation}
  R(f^{\prime}) = \left| \frac{1}{2} f^{\prime\prime}(x) h \right| \overset{!}{=} \symcal{O}(h)
\end{equation}
gegeben.

Eine verbesserte Variante stellt der Differenzenquotient mit zwei symmetrischen Punkten $x ± h$ dar:
\begin{equation}
 f^{\prime}(x) = \frac{f(x + h) - f(x - h)}{2h} + \symcal{O}(h²),
\end{equation}
mit dem Abbruchfehler
\begin{equation}
  R(f^{\prime}) = \left| \frac{1}{6} f^{\prime\prime\prime}(x) h² \right| \overset{!}{=} \symcal{O}(h²).
\end{equation}

Eine noch aufwendigere Formel ist die symmetrische Vierpunkt-Formel:
\begin{equation}
  f^{\prime} (x) = \frac{f(x - 2h) - 8f(x - h) + 8f(x + h) - f(x + 2h)}{12 h} + \symcal{O}(h⁴),
\end{equation}
welche jedoch aufgrund der Auswertung von $f(x)$ an vier Punkten kaum Anwendung findet in der Praxis.

Für die zweite Ableitung erhält man als doppelten Differenzenquotienten:
\begin{equation}
  f^{\prime\prime}(x) = \frac{f(x + h) - 2f(x) + f(x - h)}{h²} + \symcal{O}(h²).
\end{equation}

\begin{enumerate}[(i)]
\item Implementieren Sie einen abstrakten Datentypen \mintinline{fortran}{basic_func_t}, welcher eine deferred Typ-gebundene Prozedure \mintinline{fortran}{evaluate} besitzt, welche als Funktion ein Argument $x$ vom Datentyp \mintinline{fortran}{real} und \mintinline{fortran}{intent(in)} erhält und als Rückgabewert $y$ vom Datentyp \mintinline{fortran}{real} zurückgibt.
\item Implementieren Sie als Erweiterung des abstrakten Datentyp \mintinline{fortran}{basic_func_t} die Funktion $f(x) = ax^n$ mit den beiden Parameter $a$ und $n$ als Datentyp \mintinline{fortran}{polynom_func_t}, welche als Datenfelder $a$ und $n$ vom Datentyp \mintinline{fortran}{real} besitzt.
\item Implementieren Sie für den abgeleiteten Datentyp \mintinline{fortran}{polynom_func_t} eine Typ-gebundene Prozedure \mintinline{fortran}{init} und \mintinline{fortran}{write} als Subroutinen, sowie die deferred Typ-gebundene Prozedure \mintinline{fortran}{evaluate}
  Die Prozedure \mintinline{fortran}{init} übernimmt als Argumente $a$ und $n$ und setzt diese in den entsprechenden Datenfelder ein.
  Die Prozedure \mintinline{fortran}{write} hat keine Argumente, sondern schreibt nur die beiden Parameter $a$ und $n$ auf die Standardausgabe.
\item Schreiben Sie ein Programm, welches die naive Ableitung, die Ableitung mit den zwei symmetrischen Punkten und den vier symmetrischen Punkten der Funktion $f(x) = 0.5 x³$ implementiert als Datentyp \mintinline{fortran}{polynom_func_t} bestimmt.
  Plotten Sie $f^{\prime}(1)$ gegen $h$, variieren Sie hierfür $h$ logarithmisch im Bereich von $[10^{-7}, 0)$.
\item Implementieren Sie den doppelten Differenzenquotienten ähnlich.
\end{enumerate}
\end{question}

\begin{question}[subtitle=Harmonischer Oszillator]
  Die Wellenfunktionen des harmonischen Oszillators lauten
  \begin{equation}
   ψ_n(x) = \frac{1}{\sqrt{2^n n! \sqrt{π}}} \exp \left( -\frac{1}{2}x² \right) H_n(x),
  \end{equation}
  mit den Hermite-Polynomen $H_n$.
  \begin{enumerate}[(i)]
  \item Schreiben Sie ein Programm zur Erzeugung der Hermite-Polynome $H_n(x)$ aus der Rekursionsbeziehung
    \begin{align}
      H₀ (x) & = 1,\\
      H₁ (x) & = 2x, \\
      H_{n + 1} (x) & = 2x H_n(x) - 2n H_{n - 1} (x).
    \end{align}
    Plotten Sie einige Hermite-Polynome $H_n$ und die Wellenfunktion $ψ_n$ für $
    = 1, 2, 5, 42$.
  \item Überprüfen Sie die Differentialgleichung, welche die Hermite-Polynome lösen,
    \begin{equation}
     H^{\prime\prime}_{n} (x) - 2x H^{\prime}_n(x) + 2nH_n(x) = 0,
   \end{equation}
   mittels numerischer Differentiation für $n = 1, 2, 5$.
  \end{enumerate}
  \textit{Hinweis}: Bestimmen Sie vorab die Koeffizienten $a_{n, k}$ des Hermite-Polynom $\smash{H_{n}(x) = ∑_{k = 0}^n a_{n, k} x^k}$ aus der Rekursionsformel. Inbesondere, werwenden Sie 64-bit Floating-Point- und Integer-Zahlen, sowie die Kennzeichung \mintinline{fortran}{recursive} für sich selbstaufrufende Prozeduren.
\end{question}
\end{document}