\documentclass{scrartcl}

% Paket float verbessern
\usepackage{scrhack}

% Kleine Anpassung von maketitle
\usepackage{titling}
% \usepackage{showframe}

% Warnung, falls nochmal kompiliert werden muss
\usepackage[aux]{rerunfilecheck}

% unverzichtbare Mathe-Befehle
\usepackage{amsmath}
% viele Mathe-Symbole
\usepackage{amssymb}
% Erweiterungen für amsmath
\usepackage{mathtools}

% Fonteinstellungen
\usepackage{selnolig}
\usepackage{fontspec}
% Latin Modern Fonts werden automatisch geladen
% Alternativ zum Beispiel:
%\setromanfont{Libertinus Serif}
%\setsansfont{Libertinus Sans}
%\setmonofont{Libertinus Mono}

% Wenn man andere Schriftarten gesetzt hat,
% sollte man das Seiten-Layout neu berechnen lassen
\recalctypearea{}

% deutsche Spracheinstellungen
\usepackage{polyglossia}
\setmainlanguage{german}
\setotherlanguage{english}

\usepackage[
  math-style=ISO,    % ┐
  bold-style=ISO,    % │
  sans-style=italic, % │ ISO-Standard folgen
  nabla=upright,     % │
  partial=upright,   % ┘
  warnings-off={           % ┐
    mathtools-colon,       % │ unnötige Warnungen ausschalten
    mathtools-overbracket, % │
  },                       % ┘
]{unicode-math}

% traditionelle Fonts für Mathematik
\setmathfont{Latin Modern Math}
% Alternativ zum Beispiel:
%\setmathfont{Libertinus Math}

\setmathfont{XITS Math}[range={scr, bfscr}]
\setmathfont{XITS Math}[range={cal, bfcal}, StylisticSet=1]

% Zahlen und Einheiten
\usepackage[
  locale=DE,                   % deutsche Einstellungen
  separate-uncertainty=true,   % immer Fehler mit \pm
  per-mode=symbol-or-fraction, % / in inline math, fraction in display math
]{siunitx}

% chemische Formeln
\usepackage[
  version=4,
  math-greek=default, % ┐ mit unicode-math zusammenarbeiten
  text-greek=default, % ┘
]{mhchem}

% richtige Anführungszeichen
\usepackage[autostyle]{csquotes}

% schöne Brüche im Text
\usepackage{xfrac}

% Standardplatzierung für Floats einstellen
\usepackage{float}
\floatplacement{figure}{htbp}
\floatplacement{table}{htbp}

% Floats innerhalb einer Section halten
\usepackage[
  section, % Floats innerhalb der Section halten
  below,   % unterhalb der Section aber auf der selben Seite ist ok
]{placeins}

% Seite drehen für breite Tabellen: landscape Umgebung
\usepackage{pdflscape}

% Captions schöner machen.
\usepackage[
  labelfont=bf,        % Tabelle x: Abbildung y: ist jetzt fett
  font=small,          % Schrift etwas kleiner als Dokument
  width=0.9\textwidth, % maximale Breite einer Caption schmaler
]{caption}
% subfigure, subtable, subref
\usepackage{subcaption}

% Grafiken können eingebunden werden
\usepackage{graphicx}
% größere Variation von Dateinamen möglich
\usepackage{grffile}

% schöne Tabellen
\usepackage{booktabs}

% Verbesserungen am Schriftbild
\usepackage{microtype}

% Literaturverzeichnis
\usepackage[
  backend=biber,
]{biblatex}
% Quellendatenbank
% \addbibresource{lit.bib}
% \addbibresource{programme.bib}

% Hyperlinks im Dokument
\usepackage[
  unicode,        % Unicode in PDF-Attributen erlauben
  pdfusetitle,    % Titel, Autoren und Datum als PDF-Attribute
  pdfcreator={},  % ┐ PDF-Attribute säubern
  pdfproducer={}, % ┘
]{hyperref}
% erweiterte Bookmarks im PDF
\usepackage{bookmark}
\usepackage{cleveref} % muss nach hyperref kommen

% Trennung von Wörtern mit Strichen
\usepackage[shortcuts]{extdash}

\usepackage{expl3}
\usepackage{xparse}

\usepackage{minted}
\usepackage{enumerate}
\usepackage{exsheets}
\SetupExSheets{
  headings=block-subtitle,
  question/name=Aufgabe,
}

\DeclareInstance{exsheets-heading}{block-subtitle}{default}{
  join = {
    title[r,B]number[l,B](.333em,0pt) ;
    title[r,B]subtitle[l,B](1em,0pt)
  } ,
  attach = {
    main[l,vc]title[l,vc](0pt,0pt) ;
    main[r,vc]points[l,vc](\marginparsep,0pt)
  }
}

\setlength{\droptitle}{-4em}
\pretitle{\begin{center}\Large}
\posttitle{\par\end{center}}
\preauthor{
  \begin{center}\large \lineskip 0.2em
    \begin{tabular}[t]{c}
}
\postauthor{\end{tabular}\par\end{center}}
\predate{\begin{center}\normalsize}
\postdate{\par\end{center}}

\setlength{\parindent}{0cm}
\setlength{\topmargin}{0in}
\setlength{\headheight}{0cm}
\setlength{\headsep}{0cm}
\setlength{\textheight}{9.0in}
\setlength{\evensidemargin}{0.0in}
\setlength{\oddsidemargin}{0.0in}
\setlength{\textwidth}{6in}
\setlength{\footskip}{.8in}

\title{\textsf{\textbf{Übungen zum wissenschaftlichen Programmieren SS2019}}}
\author{Prof.\,Dr.\,W.\,Kilian\unskip, S.\,Braß}

\ExplSyntaxOn
\RenewDocumentCommand \maketitlehooka {} {
  \rule{\textwidth}{1pt}
}
\RenewDocumentCommand \maketitlehookd {} {
  \rule{\textwidth}{1pt}
}
\RenewDocumentCommand \maketitlehookc {} {
  \begin{center}
    \large \AufgabenBlatt
  \end{center}
}
\NewDocumentCommand \I {}
{
\symup{i}
}
\NewDocumentCommand \Sp {}
{
  \operatorname{Sp}
}
\NewDocumentCommand \adj {}
{
  \operatorname{adj}
}
\ExplSyntaxOff

\NewDocumentCommand\AufgabenBlatt{}{Übungsblatt 6}
\date{Ausgabe: Fr, 16.05.2019, Besprechung: Fr, 21.05.2019}
\setcounter{question}{11}
\begin{document}

\maketitle

\begin{question}[subtitle=Numerische Integration]

  Die Integration ist ähnlich wie die Differentiation über einen Grenzprozess, z.\,Bsp. als Grenzwert der Riemann-Summe, definiert,
  \begin{equation}
    ∫_a^b f(x)\, \symup{d} x = \lim_{h → 0} ∑_{k = 1}^{N(h)} h f(xₖ) \quad \text{mit } xₖ = a + kh \text{ und } N(h) = \frac{b - a}{h},
  \end{equation}
  der so jedoch im Computer nicht ausgeführt werden kann.
  Zur numerischen Näherung des Integrals geht man daher folgendermaßen vor:
  \begin{enumerate}
  \item Das Intervall $[a, b]$ wird in $N$ Teilintervalle der Länge $h = \sfrac{b - a}{N}$ zerlegt.
  \item Das Integral $∫_{xₖ}^{xₖ + h} f(x) \, \symup{d}x$ wird im Teilintervall $k$ mithilfe einer Taylor-Entwicklung des Integranden zur $n$-ten Ordnung berechnet.
  \item Der Fehler der resultierenden Integrationsformel kann systematisch mit der Euler--McLaurin Formel abgeschätzt werden.
  \end{enumerate}
  Die verschiedenen Integrationsregel unterscheiden sich in der Wahl der Stützstellen $xₖ$, sowie der Ordnung der Taylor-Entwicklung und ihrer Fehlerabschätzung.
  \begin{description}
  \item[Trapezregel]
    Für die Trapezregel wird der Integrand zu ersten Ordnungn genähert und am Rand der Teilintervalle ausgewertet.
    Die Fehlerabschätzung hängt u.a.\ von der Anzahl der Teilintervalle ab und ist von der Ordnung $\symcal{O}(N^{-2})$.
    \begin{equation}
      ∫_a^b f(x) \,\symup{d}x = h ∑_{k = 1}^{N - 1} f(xₖ) + \frac{h}{2} \left( f(a) + f(b) \right) + \symcal{O}(N^{-2}), \quad \text{mit} xₖ = a + kh, N(h) = \frac{b - a}{h}.
    \end{equation}
  \item[Mittelpunktsregel]
    Für die Mittelpunktsregel wird der Integrand zur ersten Ordnung genähert und in der Mitte der Teilintervalle ausgewertet. Die Fehlerabschätzung ist on der Ordnung $\symcal{O}(N^{-2})$,
    \begin{equation}
      ∫_a^b f(x) \, \symup{d}x = h ∑^{N}_{k = 1} f(a - \frac{h}{2} + kh) + \symcal{O}(N^{-2}).
    \end{equation}
  \item[Simpsonregel]
    Für die Simpsonregel wird das Intervall $[a, b]$ in $\sfrac{N}{2}$ nebeneinanderliegende und gleich große Teilintervalle $[x_{k}, x_{k + 2}]$ mit Mittelpunkt $xₖ = \sfrac{(x_{k + 2} - x_{k})}{2}$ und Länge $2h$ zerlegt.
    Der Integrand wird zur zweiten Ordnung entwickelt mit der Fehlerabschätzung zur Ordnung $\symcal{O}(N^{-4})$.
    Damit ergibt sich dann,
    \begin{equation}
      ∫_a^b f(x) \, \symup{d}x = \frac{h}{3}\left( f(a)+ 2 \sum_{k=1}^{N-1}f(x_{2k})+ f(x_{2n}) + 4\sum_{k=1}^{N}f(x_{2k-1}) \right) + \symcal{O}(N^{-4}).
    \end{equation}
  \end{description}
  \begin{enumerate}[(i)]
  \item Implementieren Sie jeweils eine Integrationsroutine für die Trapez-, die Simpson- und die Mittelpunktsregel, and die folgende vier Argumente übergeben werden sollen: (1) Integrand f(x) als abstrakter Datentyp \mintinline{fortran}{basic_func_t}, (2) untere Grenze $a$, (3) obere Grenze $b$ und (4) Integrationsintervallbreite $h$ oder Zahl der Integrationsintervalle $N$, welcher bei der Simpsonregel gerade sein sollte.
  \item Berechnen Sie folgende Integrale numerisch, jeweils mittels (i) Trapezregel, (ii) Simpsonregel. Halbieren Sie bei beiden Verfahren die Intervallbreite $h$ bis die relative Änderung des Ergebnisses kleiner als $10^{-4}$ wird.
    \begin{enumerate}[a)]
    \item
      \begin{equation}
        I₁ = ∫_{1}^{100} \symup{d}x\, \frac{\exp{-x}}{x}
      \end{equation}
    \item
      \begin{equation}
        I₂ = ∫_{0}^{1} \symup{d}x\, x\sin \left( \frac{1}{x} \right)
      \end{equation}
    \end{enumerate}
    Zur Kontrolle: $I₁ ≈ \num{0,21938}, I₂ ≈ 0.37853$.
  \end{enumerate}
\end{question}

\begin{question}[subtitle=Uneigentliche und Hauptwert-Integrale]
  \begin{enumerate}[(i)]
  \item Berechnen Sie folgendes Hauptwertintegral numerisch:
    \begin{equation}
      I₁ = \symcal{P} ∫_{-1}^{1} \! \symup{d}t\, \frac{\exp{t}}{t}.
    \end{equation}
  \item Berechnen Sie folgendes Integral numerisch mit einem relativen Fehler $ε ≤ 10^{-5}$:
    \begin{equation}
      I₂ = ∫_0^∞ \! \symup{d}t \, \frac{\exp{-t}}{\sqrt{t}}.
    \end{equation}
    Berechnen Sie das Integral zum Vergleich analytisch.
  \end{enumerate}
  Zur Kontrolle: $I₁ ≈ \num{2.1145018}, I₂ ≈ \num{1.77245385}$.
\end{question}
\end{document}