\documentclass{scrartcl}

% Paket float verbessern
\usepackage{scrhack}

% Kleine Anpassung von maketitle
\usepackage{titling}
% \usepackage{showframe}

% Warnung, falls nochmal kompiliert werden muss
\usepackage[aux]{rerunfilecheck}

% unverzichtbare Mathe-Befehle
\usepackage{amsmath}
% viele Mathe-Symbole
\usepackage{amssymb}
% Erweiterungen für amsmath
\usepackage{mathtools}

% Fonteinstellungen
\usepackage{selnolig}
\usepackage{fontspec}
% Latin Modern Fonts werden automatisch geladen
% Alternativ zum Beispiel:
%\setromanfont{Libertinus Serif}
%\setsansfont{Libertinus Sans}
%\setmonofont{Libertinus Mono}

% Wenn man andere Schriftarten gesetzt hat,
% sollte man das Seiten-Layout neu berechnen lassen
\recalctypearea{}

% deutsche Spracheinstellungen
\usepackage{polyglossia}
\setmainlanguage{german}
\setotherlanguage{english}

\usepackage[
  math-style=ISO,    % ┐
  bold-style=ISO,    % │
  sans-style=italic, % │ ISO-Standard folgen
  nabla=upright,     % │
  partial=upright,   % ┘
  warnings-off={           % ┐
    mathtools-colon,       % │ unnötige Warnungen ausschalten
    mathtools-overbracket, % │
  },                       % ┘
]{unicode-math}

% traditionelle Fonts für Mathematik
\setmathfont{Latin Modern Math}
% Alternativ zum Beispiel:
%\setmathfont{Libertinus Math}

\setmathfont{XITS Math}[range={scr, bfscr}]
\setmathfont{XITS Math}[range={cal, bfcal}, StylisticSet=1]

% Zahlen und Einheiten
\usepackage[
  locale=DE,                   % deutsche Einstellungen
  separate-uncertainty=true,   % immer Fehler mit \pm
  per-mode=symbol-or-fraction, % / in inline math, fraction in display math
]{siunitx}

% chemische Formeln
\usepackage[
  version=4,
  math-greek=default, % ┐ mit unicode-math zusammenarbeiten
  text-greek=default, % ┘
]{mhchem}

% richtige Anführungszeichen
\usepackage[autostyle]{csquotes}

% schöne Brüche im Text
\usepackage{xfrac}

% Standardplatzierung für Floats einstellen
\usepackage{float}
\floatplacement{figure}{htbp}
\floatplacement{table}{htbp}

% Floats innerhalb einer Section halten
\usepackage[
  section, % Floats innerhalb der Section halten
  below,   % unterhalb der Section aber auf der selben Seite ist ok
]{placeins}

% Seite drehen für breite Tabellen: landscape Umgebung
\usepackage{pdflscape}

% Captions schöner machen.
\usepackage[
  labelfont=bf,        % Tabelle x: Abbildung y: ist jetzt fett
  font=small,          % Schrift etwas kleiner als Dokument
  width=0.9\textwidth, % maximale Breite einer Caption schmaler
]{caption}
% subfigure, subtable, subref
\usepackage{subcaption}

% Grafiken können eingebunden werden
\usepackage{graphicx}
% größere Variation von Dateinamen möglich
\usepackage{grffile}

% schöne Tabellen
\usepackage{booktabs}

% Verbesserungen am Schriftbild
\usepackage{microtype}

% Literaturverzeichnis
\usepackage[
  backend=biber,
]{biblatex}
% Quellendatenbank
% \addbibresource{lit.bib}
% \addbibresource{programme.bib}

% Hyperlinks im Dokument
\usepackage[
  unicode,        % Unicode in PDF-Attributen erlauben
  pdfusetitle,    % Titel, Autoren und Datum als PDF-Attribute
  pdfcreator={},  % ┐ PDF-Attribute säubern
  pdfproducer={}, % ┘
]{hyperref}
% erweiterte Bookmarks im PDF
\usepackage{bookmark}
\usepackage{cleveref} % muss nach hyperref kommen

% Trennung von Wörtern mit Strichen
\usepackage[shortcuts]{extdash}

\usepackage{expl3}
\usepackage{xparse}

\usepackage{minted}
\usepackage{enumerate}
\usepackage{exsheets}
\SetupExSheets{
  headings=block-subtitle,
  question/name=Aufgabe,
}

\DeclareInstance{exsheets-heading}{block-subtitle}{default}{
  join = {
    title[r,B]number[l,B](.333em,0pt) ;
    title[r,B]subtitle[l,B](1em,0pt)
  } ,
  attach = {
    main[l,vc]title[l,vc](0pt,0pt) ;
    main[r,vc]points[l,vc](\marginparsep,0pt)
  }
}

\setlength{\droptitle}{-4em}
\pretitle{\begin{center}\Large}
\posttitle{\par\end{center}}
\preauthor{
  \begin{center}\large \lineskip 0.2em
    \begin{tabular}[t]{c}
}
\postauthor{\end{tabular}\par\end{center}}
\predate{\begin{center}\normalsize}
\postdate{\par\end{center}}

\setlength{\parindent}{0cm}
\setlength{\topmargin}{0in}
\setlength{\headheight}{0cm}
\setlength{\headsep}{0cm}
\setlength{\textheight}{9.0in}
\setlength{\evensidemargin}{0.0in}
\setlength{\oddsidemargin}{0.0in}
\setlength{\textwidth}{6in}
\setlength{\footskip}{.8in}

\title{\textsf{\textbf{Übungen zum wissenschaftlichen Programmieren SS2019}}}
\author{Prof.\,Dr.\,W.\,Kilian\unskip, S.\,Braß}

\ExplSyntaxOn
\RenewDocumentCommand \maketitlehooka {} {
  \rule{\textwidth}{1pt}
}
\RenewDocumentCommand \maketitlehookd {} {
  \rule{\textwidth}{1pt}
}
\RenewDocumentCommand \maketitlehookc {} {
  \begin{center}
    \large \AufgabenBlatt
  \end{center}
}
\NewDocumentCommand \I {}
{
\symup{i}
}
\NewDocumentCommand \Sp {}
{
  \operatorname{Sp}
}
\NewDocumentCommand \adj {}
{
  \operatorname{adj}
}
\ExplSyntaxOff

\NewDocumentCommand\AufgabenBlatt{}{Übungsblatt 7}
\date{Ausgabe: Fr, 24.05.2019, Besprechung: Fr, 14.06.2019}
\setcounter{question}{13}
\begin{document}

\maketitle

Die bisherigen Integrationsregeln sind für ein-dimensionale Integrale definiert worden.
In der Physik treten jedoch meistens mehrdimensionale Integrale auf.
Um mehrdimensionale Integrale zu berechnen, kann das mehrdimensionale Integral durch die hintereinander Ausführung von ein-dimensionalen Integrale nach dem Satz von Fubini berechnet werden,
\begin{equation}
  ∫_{[a, b]^n} f(x₁, …, xₙ) \, \symup{d}^n \symbf{x} =  ∫_{a}^b \symup{d}x₁ … ∫_{a}^{b}\symup{d}xₙ f(x₁,…, xₙ).
\end{equation}
Der Nachteil bei diesem Ansatz ist, dass die Anzahl der Stützstellen exponentiell mit der Anzahl der Dimensionen anwächst; der sogenannte Fluch der Dimensionalität.

\begin{question}[subtitle=Beugung]
  Berechnen Sie numerisch die Intensität bei Fraunhoferbeugung an einem Ring $a < |\symbf{r}| < 2a$:
  \begin{align}
    I(\symbf{q}) & = C |u(\symbf{q})|², \\
    u(\symbf{q}) & = ∫_{\text{Ring}} \symup{d}²\symbf{r} \, e^{-i \symbf{q}·\symbf{r}}.
  \end{align}
  \begin{enumerate}[(i)]
  \item Schreiben Sie das Integral in Polarkoordinaten.
    Warum hängt $u$ nur von $q ≡ |\symbf{q}|$ ab? Schreiben Sie $u(q)$ als reelles Integral und berechnen Sie $u(q)$ und $I(q)$ analytisch.
  \item Führen Sie das zwei-dimensionale Integral numerisch in Polarkoordinaten (Winkel- und Radialintegration) für $q = 0.1 \sfrac{a}{n}$ mit $n = 0, 1, 2, …, 50$ aus.
    Plotten Sie die entsprechenden Werte $I(q)$ (mit $C = 1$).
  \item Berechnen Sie die Intensität $I(\symbf{q})$ auch für einen Viertelring $a < |\symbf{r}| < 2a$ und $0 < ϕ < \sfrac{π}{2}$ (in Polarkoordinaten), indem Sie das zwei-dimensionale Integral wieder numerisch in Polarkoordinaten ausführen.
    Werten Sie dazu Realteil und Imaginärteil von $u(qₓ, q_y)$ getrennt aus, z.\,Bsp.\ für $qₓ = 0.2 \sfrac{n}{a}$ und $q_y = 0.2 \sfrac{m}{a}$ mit $m, n = -30, …, 0, … 30$.
  \end{enumerate}
\end{question}

\begin{question}[subtitle=Elektrostatik]
  Wir berechnen das elektrostatische Potential für zwei Ladungsverteilungen in einem Würfel der Kantenlänge $2a$ numerisch durch direkte drei-dimensionale Integration.
  \begin{enumerate}[(i)]
  \item Zuerst betrachten wir eine homogene Ladungsverteilung
    \begin{equation}
      ρ(\symbf{r}) =
      \begin{dcases*}
        ρ₀ & $|x| < a, |y| < a, |z| < a$ \\
        0 & sonst
      \end{dcases*},
    \end{equation}
    und berechnen das elektrostatische Potential auf der $x$-Achse
    \begin{equation}
      ϕ(x) = \frac{1}{4πε₀} ∫\symup{d}³\symbf{r}^\prime \, \frac{ρ(\symbf{r}^\prime)}{(x \symbf{e}_x - \symbf{r}^\prime)^{\sfrac{3}{2}}}.
    \end{equation}
    Führen Sie eine geeignete Wahl der Einheiten für $ϕ$ und $x$ ein, um das numerisch zu berechnende Integral einheitenlos zu machen.

    Führen Sie dann das drei-dimensionale Integral numerisch aus, zunächst für $x$-Werte $\sfrac{x}{a} = 0.1 n$ mit $n = 11, 12, …, 80$ außerhalb des Würfels.
    Plotten Sie die entsprechenden Werte $ϕ(x)$.
    Welche Asymptotik erwarten Sie für große $x$? (Multipolentwicklung)
    Überprüfen Sie Ihre Vermutung.
  \item Versuchen Sie das Integral auch für $|x| < a$ innerhalb des Würfels ($x$-Werte $\sfrac{x}{a} = 0.1 n$ mit $n = 0, 2, …, 10$) numerisch auszuwerten.
    Könnte es Probleme geben?
  \item Betrachten Sie die Ladungsverteilung
    \begin{equation}
      ρ(\symbf{r}) =
      \begin{dcases*}
        ρ₀ \sfrac{x}{a} & $|x| < a, |y| < a, |z| < a$ \\
        0 & sonst
      \end{dcases*}.
    \end{equation}
    Überlegen Sie sich wieder, wie Sie das numerisch zu berechnende Integral einheitenlos machen.

    Führen Sie dann wieder das drei-dimensionale Integral numerisch aus, außerhalb und innerhalb des Würfels, d.\,h. für $x$-Werte $\sfrac{x}{a} = 0.1 n$ mit $n = 0, 1, 2, …, 80$.
    Plotten Sie die entsprechenden Werte $ϕ(x)$.
    Welche Asymptotik erwarten Sie nun für große $x$?
    Berechnen Sie das erste nicht-verschwindende Multiplotmoment und überprüfen Sie das zu erwartende asymptotische Verhalten.
  \end{enumerate}
\end{question}
\end{document}