\documentclass{scrartcl}

% Paket float verbessern
\usepackage{scrhack}

% Kleine Anpassung von maketitle
\usepackage{titling}
% \usepackage{showframe}

% Warnung, falls nochmal kompiliert werden muss
\usepackage[aux]{rerunfilecheck}

% unverzichtbare Mathe-Befehle
\usepackage{amsmath}
% viele Mathe-Symbole
\usepackage{amssymb}
% Erweiterungen für amsmath
\usepackage{mathtools}

% Fonteinstellungen
\usepackage{selnolig}
\usepackage{fontspec}
% Latin Modern Fonts werden automatisch geladen
% Alternativ zum Beispiel:
%\setromanfont{Libertinus Serif}
%\setsansfont{Libertinus Sans}
%\setmonofont{Libertinus Mono}

% Wenn man andere Schriftarten gesetzt hat,
% sollte man das Seiten-Layout neu berechnen lassen
\recalctypearea{}

% deutsche Spracheinstellungen
\usepackage{polyglossia}
\setmainlanguage{german}
\setotherlanguage{english}

\usepackage[
  math-style=ISO,    % ┐
  bold-style=ISO,    % │
  sans-style=italic, % │ ISO-Standard folgen
  nabla=upright,     % │
  partial=upright,   % ┘
  warnings-off={           % ┐
    mathtools-colon,       % │ unnötige Warnungen ausschalten
    mathtools-overbracket, % │
  },                       % ┘
]{unicode-math}

% traditionelle Fonts für Mathematik
\setmathfont{Latin Modern Math}
% Alternativ zum Beispiel:
%\setmathfont{Libertinus Math}

\setmathfont{XITS Math}[range={scr, bfscr}]
\setmathfont{XITS Math}[range={cal, bfcal}, StylisticSet=1]

% Zahlen und Einheiten
\usepackage[
  locale=DE,                   % deutsche Einstellungen
  separate-uncertainty=true,   % immer Fehler mit \pm
  per-mode=symbol-or-fraction, % / in inline math, fraction in display math
]{siunitx}

% chemische Formeln
\usepackage[
  version=4,
  math-greek=default, % ┐ mit unicode-math zusammenarbeiten
  text-greek=default, % ┘
]{mhchem}

% richtige Anführungszeichen
\usepackage[autostyle]{csquotes}

% schöne Brüche im Text
\usepackage{xfrac}

% Standardplatzierung für Floats einstellen
\usepackage{float}
\floatplacement{figure}{htbp}
\floatplacement{table}{htbp}

% Floats innerhalb einer Section halten
\usepackage[
  section, % Floats innerhalb der Section halten
  below,   % unterhalb der Section aber auf der selben Seite ist ok
]{placeins}

% Seite drehen für breite Tabellen: landscape Umgebung
\usepackage{pdflscape}

% Captions schöner machen.
\usepackage[
  labelfont=bf,        % Tabelle x: Abbildung y: ist jetzt fett
  font=small,          % Schrift etwas kleiner als Dokument
  width=0.9\textwidth, % maximale Breite einer Caption schmaler
]{caption}
% subfigure, subtable, subref
\usepackage{subcaption}

% Grafiken können eingebunden werden
\usepackage{graphicx}
% größere Variation von Dateinamen möglich
\usepackage{grffile}

% schöne Tabellen
\usepackage{booktabs}

% Verbesserungen am Schriftbild
\usepackage{microtype}

% Literaturverzeichnis
\usepackage[
  backend=biber,
]{biblatex}
% Quellendatenbank
% \addbibresource{lit.bib}
% \addbibresource{programme.bib}

% Hyperlinks im Dokument
\usepackage[
  unicode,        % Unicode in PDF-Attributen erlauben
  pdfusetitle,    % Titel, Autoren und Datum als PDF-Attribute
  pdfcreator={},  % ┐ PDF-Attribute säubern
  pdfproducer={}, % ┘
]{hyperref}
% erweiterte Bookmarks im PDF
\usepackage{bookmark}
\usepackage{cleveref} % muss nach hyperref kommen

% Trennung von Wörtern mit Strichen
\usepackage[shortcuts]{extdash}

\usepackage{expl3}
\usepackage{xparse}

\usepackage{minted}
\usepackage{enumerate}
\usepackage{exsheets}
\SetupExSheets{
  headings=block-subtitle,
  question/name=Aufgabe,
}

\DeclareInstance{exsheets-heading}{block-subtitle}{default}{
  join = {
    title[r,B]number[l,B](.333em,0pt) ;
    title[r,B]subtitle[l,B](1em,0pt)
  } ,
  attach = {
    main[l,vc]title[l,vc](0pt,0pt) ;
    main[r,vc]points[l,vc](\marginparsep,0pt)
  }
}

\setlength{\droptitle}{-4em}
\pretitle{\begin{center}\Large}
\posttitle{\par\end{center}}
\preauthor{
  \begin{center}\large \lineskip 0.2em
    \begin{tabular}[t]{c}
}
\postauthor{\end{tabular}\par\end{center}}
\predate{\begin{center}\normalsize}
\postdate{\par\end{center}}

\setlength{\parindent}{0cm}
\setlength{\topmargin}{0in}
\setlength{\headheight}{0cm}
\setlength{\headsep}{0cm}
\setlength{\textheight}{9.0in}
\setlength{\evensidemargin}{0.0in}
\setlength{\oddsidemargin}{0.0in}
\setlength{\textwidth}{6in}
\setlength{\footskip}{.8in}

\title{\textsf{\textbf{Übungen zum wissenschaftlichen Programmieren SS2019}}}
\author{Prof.\,Dr.\,W.\,Kilian\unskip, S.\,Braß}

\ExplSyntaxOn
\RenewDocumentCommand \maketitlehooka {} {
  \rule{\textwidth}{1pt}
}
\RenewDocumentCommand \maketitlehookd {} {
  \rule{\textwidth}{1pt}
}
\RenewDocumentCommand \maketitlehookc {} {
  \begin{center}
    \large \AufgabenBlatt
  \end{center}
}
\NewDocumentCommand \I {}
{
\symup{i}
}
\NewDocumentCommand \Sp {}
{
  \operatorname{Sp}
}
\NewDocumentCommand \adj {}
{
  \operatorname{adj}
}
\ExplSyntaxOff

\NewDocumentCommand\AufgabenBlatt{}{Übungsblatt 8}
\date{Ausgabe: Di, 18.06.2019, Besprechung: Fr, 28.06.2019}
\setcounter{question}{16}

\DeclareMathOperator{\odeg}{deg}

\begin{document}

\maketitle

\begin{question}[subtitle=Polynome und Unittests]
  Im folgenden betrachten wir Polynome $p(x)$ vom Grad $n$ mit reellen Koeffizienten $aᵢ$,
  \begin{equation}
    p(x) = ∑_{i = 0}^n aᵢ x^i, n ≥ 0.
  \end{equation}
  \begin{enumerate}[(i)]
  \item Implementieren Sie einen Datentyp \mintinline{fortran}{polynom_t} in einem eigenen Modul, welcher folgende typ-gebundene Prozeduren besitzt:
    \begin{itemize}
    \item \mintinline{fortran}{init}: Initialisieren mit einem Array von Koeffizienten, bzw. mit dem führenden Koeffizienten $aₙ$ sowie dem Grad $n$ und (implizit) $a_{i ≠ n} = 0$,
    \item \mintinline{fortran}{write}: Formatierte Ausgabe des Polynoms,
    \item \mintinline{fortran}{lc}: Rückgabe des führenden Koeffizienten $a_n$,
    \item \mintinline{fortran}{deg}: Rückgabe des Grad $n$,
    \item \mintinline{fortran}{get_coeff}: Rückgabe der Koeffizienten $a_i$ als $(n+1)$-Array,
    \item \mintinline{fortran}{operator(+)}: Addition zweier Polynome von beliebigen Graden $n, m$,
    \item \mintinline{fortran}{operator(-)}: Subtraktion zweier Polynome von beliebigen Graden $n, m$,
    \item \mintinline{fortran}{operator(*)}: Multiplikation zweier Polynome von beliebigen Graden $n, m$,
    \item \mintinline{fortran}{reduce}: Reduzieren des Grades des Polynomes bis zum ersten nicht-verschwindenden führenden Koeffizienten.
    \end{itemize}
  \item Testen Sie jede typ-gebundene Prozedur mit geeigneten Unittests nach dem Prinzip: Vorbereiten, Durchführen und Bestätigen.

    Schreiben Sie für jede typ-gebundene Prozedur einen Subroutine, welche als Argument einen logischen Wert über den Erfolg des Tests zurückgibt.
    Die Prozedure initialisiert ein (oder mehrere) Objekt(e) vom Typ \mintinline{fortran}{polynom_t} mit entsprechenden Koeffizienten, führt eine entsprechende Aktion durch und gleicht das Ergebnis ab.
    Überlegen Sie sich für den Abgleich, wie Sie Fließkommazahlen (einer bestimmten Präzision) sinnvoll vergleichen können.
  \end{enumerate}
\end{question}

\begin{question}[subtitle=Euklidische Polynomdivision]
  Im folgenden sei die euklidische Polynomdivision gegeben:\\
  Seien $a, b$ zwei reelle Polynome in einer Variable, $b ≠ 0$ und $\odeg(a) ≥ \odeg(b)$, wobei $\odeg$ den Grad eines Polynoms gibt.

  Dann existieren zwei Polynome, $q$, der Quotient, und $r$, der Rest, mit
  \begin{equation}
    \label{eq:euclidian-division}
    a = b \cdot q + r, \quad \odeg(r) < \odeg(b).
  \end{equation}

  Der Pseudocode für den Algorithmus zur Berechnung von $q$ und $b$ lautet:
  \inputminted{fortran}{../src/Algorithm8.f08}

  \begin{enumerate}[(i)]
  \item Implementieren Sie einen abstrakten Datentyp \mintinline{fortran}{polynom_divisor_t} in einem eigenem Modul, welcher ein Quotienten- und ein Rest-Polynom vom Typ \mintinline{fortran}{polynom_t} als \mintinline{fortran}{private} Felder enthält.
    Weiterhin, implementieren Sie folgende typ-gebundene Prozeduren: \mintinline{fortran}{write} (deferred), \mintinline{fortran}{divide} (deferred), \mintinline{fortran}{get_quotient} um das Quotienten-Polynom zu erhalten, \mintinline{fortran}{get_remainder} um das Rest-Polynom zu erhalten.
  \item Implementieren Sie als Typen-Erweiterung von \mintinline{fortran}{polynom_divisor_t} die Polynomdivision.

    \textit{Hinweis:} Greifen Sie für eine möglichst übersichtliche Implementierung der euklidischen Polynomdivision auf die überladenen Operatoren von \mintinline{fortran}{polynom_t} zurück.


  \item Testen Sie die Polynomdivision mit entsprechenden Unittests, welche die verschiedenen Szenarien der Polynomdivision prüfen.
  \end{enumerate}
\end{question}
\end{document}