\documentclass{scrartcl}

% Paket float verbessern
\usepackage{scrhack}

% Kleine Anpassung von maketitle
\usepackage{titling}
% \usepackage{showframe}

% Warnung, falls nochmal kompiliert werden muss
\usepackage[aux]{rerunfilecheck}

% unverzichtbare Mathe-Befehle
\usepackage{amsmath}
% viele Mathe-Symbole
\usepackage{amssymb}
% Erweiterungen für amsmath
\usepackage{mathtools}

% Fonteinstellungen
\usepackage{selnolig}
\usepackage{fontspec}
% Latin Modern Fonts werden automatisch geladen
% Alternativ zum Beispiel:
%\setromanfont{Libertinus Serif}
%\setsansfont{Libertinus Sans}
%\setmonofont{Libertinus Mono}

% Wenn man andere Schriftarten gesetzt hat,
% sollte man das Seiten-Layout neu berechnen lassen
\recalctypearea{}

% deutsche Spracheinstellungen
\usepackage{polyglossia}
\setmainlanguage{german}
\setotherlanguage{english}

\usepackage[
  math-style=ISO,    % ┐
  bold-style=ISO,    % │
  sans-style=italic, % │ ISO-Standard folgen
  nabla=upright,     % │
  partial=upright,   % ┘
  warnings-off={           % ┐
    mathtools-colon,       % │ unnötige Warnungen ausschalten
    mathtools-overbracket, % │
  },                       % ┘
]{unicode-math}

% traditionelle Fonts für Mathematik
\setmathfont{Latin Modern Math}
% Alternativ zum Beispiel:
%\setmathfont{Libertinus Math}

\setmathfont{XITS Math}[range={scr, bfscr}]
\setmathfont{XITS Math}[range={cal, bfcal}, StylisticSet=1]

% Zahlen und Einheiten
\usepackage[
  locale=DE,                   % deutsche Einstellungen
  separate-uncertainty=true,   % immer Fehler mit \pm
  per-mode=symbol-or-fraction, % / in inline math, fraction in display math
]{siunitx}

% chemische Formeln
\usepackage[
  version=4,
  math-greek=default, % ┐ mit unicode-math zusammenarbeiten
  text-greek=default, % ┘
]{mhchem}

% richtige Anführungszeichen
\usepackage[autostyle]{csquotes}

% schöne Brüche im Text
\usepackage{xfrac}

% Standardplatzierung für Floats einstellen
\usepackage{float}
\floatplacement{figure}{htbp}
\floatplacement{table}{htbp}

% Floats innerhalb einer Section halten
\usepackage[
  section, % Floats innerhalb der Section halten
  below,   % unterhalb der Section aber auf der selben Seite ist ok
]{placeins}

% Seite drehen für breite Tabellen: landscape Umgebung
\usepackage{pdflscape}

% Captions schöner machen.
\usepackage[
  labelfont=bf,        % Tabelle x: Abbildung y: ist jetzt fett
  font=small,          % Schrift etwas kleiner als Dokument
  width=0.9\textwidth, % maximale Breite einer Caption schmaler
]{caption}
% subfigure, subtable, subref
\usepackage{subcaption}

% Grafiken können eingebunden werden
\usepackage{graphicx}
% größere Variation von Dateinamen möglich
\usepackage{grffile}

% schöne Tabellen
\usepackage{booktabs}

% Verbesserungen am Schriftbild
\usepackage{microtype}

% Literaturverzeichnis
\usepackage[
  backend=biber,
]{biblatex}
% Quellendatenbank
% \addbibresource{lit.bib}
% \addbibresource{programme.bib}

% Hyperlinks im Dokument
\usepackage[
  unicode,        % Unicode in PDF-Attributen erlauben
  pdfusetitle,    % Titel, Autoren und Datum als PDF-Attribute
  pdfcreator={},  % ┐ PDF-Attribute säubern
  pdfproducer={}, % ┘
]{hyperref}
% erweiterte Bookmarks im PDF
\usepackage{bookmark}
\usepackage{cleveref} % muss nach hyperref kommen

% Trennung von Wörtern mit Strichen
\usepackage[shortcuts]{extdash}

\usepackage{expl3}
\usepackage{xparse}

\usepackage{minted}
\usepackage{enumerate}
\usepackage{exsheets}
\SetupExSheets{
  headings=block-subtitle,
  question/name=Aufgabe,
}

\DeclareInstance{exsheets-heading}{block-subtitle}{default}{
  join = {
    title[r,B]number[l,B](.333em,0pt) ;
    title[r,B]subtitle[l,B](1em,0pt)
  } ,
  attach = {
    main[l,vc]title[l,vc](0pt,0pt) ;
    main[r,vc]points[l,vc](\marginparsep,0pt)
  }
}

\setlength{\droptitle}{-4em}
\pretitle{\begin{center}\Large}
\posttitle{\par\end{center}}
\preauthor{
  \begin{center}\large \lineskip 0.2em
    \begin{tabular}[t]{c}
}
\postauthor{\end{tabular}\par\end{center}}
\predate{\begin{center}\normalsize}
\postdate{\par\end{center}}

\setlength{\parindent}{0cm}
\setlength{\topmargin}{0in}
\setlength{\headheight}{0cm}
\setlength{\headsep}{0cm}
\setlength{\textheight}{9.0in}
\setlength{\evensidemargin}{0.0in}
\setlength{\oddsidemargin}{0.0in}
\setlength{\textwidth}{6in}
\setlength{\footskip}{.8in}

\title{\textsf{\textbf{Übungen zum wissenschaftlichen Programmieren SS2019}}}
\author{Prof.\,Dr.\,W.\,Kilian\unskip, S.\,Braß}

\ExplSyntaxOn
\RenewDocumentCommand \maketitlehooka {} {
  \rule{\textwidth}{1pt}
}
\RenewDocumentCommand \maketitlehookd {} {
  \rule{\textwidth}{1pt}
}
\RenewDocumentCommand \maketitlehookc {} {
  \begin{center}
    \large \AufgabenBlatt
  \end{center}
}
\NewDocumentCommand \I {}
{
\symup{i}
}
\NewDocumentCommand \Sp {}
{
  \operatorname{Sp}
}
\NewDocumentCommand \adj {}
{
  \operatorname{adj}
}
\ExplSyntaxOff

\NewDocumentCommand\AufgabenBlatt{}{Übungsblatt 9}
\date{Ausgabe: Fr, 28.06.2019, Besprechung: Fr, 05.07.2019}
\setcounter{question}{18}
\begin{document}

\maketitle

Bei der Perkolation (von lat.\ \textit{percolare}: durchsickern lassen, auswaschen; motiviert durch Flüssigkeiten in porösen Medien) betrachten wir Gitter (Quadrat, Dreieck, kubisch) in $d$~Dimensionen mit $n$ Gitterplätzen (Sites) und $n_{B}$ Kanten (Bonds) zwischen den nächsten Nachbarn.
Die Zahl $n$ der Gitterplätze und $n_{B}$ hängt über $n_{B} = \frac{z}{n}$ zusammen, wobei $z$ die Koordinatenzahl, d.\,h.\ die Zahl der nächsten Nachbarn ist.
Ein kubisches Gitter der Größe $l$ hat $n = l^d$ Gitterplätze und es gilt $z = 2·d$.

Bei der Platzperkolation (Kantenperkolation) ist jeder Gitterplatz (jede Kante) entweder besetzt oder unbesetzt, wobei die Gitterplätze zufällig mit der Wahrscheinlichkeit $p$ besetzt werden.
Dadurch entstehen \textit{Cluster} von besetzten Plätzen, die durch \textquote{nächste Nachbar}-Kanten verbunden sind.

Die durchschnittliche Clustergröße wächst dabei mit $p$ an.
Die Schwellenwahrscheinlichkeit wird durch $p_c$ gegeben, dass immer ein perkolierende Cluster, welcher von einem Ende des Systems (z.\,Bsp.\ oben oder links) zum anderen Ende (z.\,B.\ unten oder rechts) reicht, auftritt.
Im thermodynamischen Limit $n, l → ∞$ ist ein perkolierender Cluster gleichbedeutend mit einem unendlichen großen Cluster.

\begin{question}[subtitle=Perkolationschwelle]
  Ziel dieser Aufgabe ist die Bestimmung von $p_c$ für die Platzperkolation auf dem Quadratgitter $(d = 2)$.
  \begin{enumerate}[(i)]
  \item Initialisierung: Erzeugen Sie eine Datenstruktur (etwa ein Array) für ein Gitter mit $l \times l$ Plätzen und erzeugen Sie $R$ Realisationen, indem Sie jeweils jeden Platz mit einer Wahrscheinlichkeit $p$ besetzen.
    Ihr Programm sollte in der Lage sein, die Systemgröße~$l$, die Besetzungswahrscheinlichkeit $p$ und die Zahl an Realisationen $R$ als Parameter zu verarbeiten.

    \begin{sloppypar}
      \textit{Hinweis:} Für die Parameterübergabe über die Kommandozeile verwenden Sie \mintinline{fortran}{command_argument_count} und \mintinline{fortran}{get_command_argument}.
    \end{sloppypar}
  \item Cluster finden: Schreiben Sie eine Prozedur, die in der Lage ist, alle Cluster aus besetzten Punkten zu finden.

    \textit{Leath}-Algorithmus: Nutzen Sie eine Liste (LIFO, FIFO), um noch abzusuchende Plätzen zu verwalten und die Nachbarn eines in den Cluster aufgenommenen Punktes hinzu zufügen.
    Die Suche startet bei einem besetzten Platz und lässt den Cluster dann solange anwachsen, bis es keine benachbarten besetzten Plätze mehr gibt.
    Das wird solange wiederholt, bis alle Plätze entweder besucht wurden oder unbesetzt sind.
    Sie sollten danach für jeden besetzten Platz wissen, in welchem Cluster er sich befindet.

    Implementieren Sie hierfür einen abgeleiteten Typ mit entsprechenden Typ-gebundenen Prozeduren für die Verwaltung und Erweiterung einer Liste.
  \item Visualisieren der Clustersuche: Plotten Sie das Ergebnis Ihrer Clustersuche für jeweils eine Realisation mit $n = 50$ für $p = 0.1, 0.5, 0.9$, z.\,B. in dem Sie die Plätze entsprechend ihrer Clusterzugehörgikeit einfärben.
  \item Überprüfen Sie, ob es einen perkolierenden Cluster gibt und bestimmen Sie die relative Perkolationshäufigkeit $q_l(p)$ für mindestens $l = 10, 50, 100$.
    Bestimmen Sie zuvor anhand von $q_{10}(p)$ einen geeigneten Wert für $R$, indem Sie $R = 10$ bis $R = 10^4$ variieren.
    % Plotten Sie die Anzahl der Cluster $n_c$ gegen die Besetzungswahrscheinlichkeit $p$ und die verschiedenen Systemgrößen $l$.
  \item Begründen Sie, warum $q_l(p_c)$ unabhängig von $l$ sein sollte, und bestimmen Sie damit $p_c$ mit möglichst hoher Genauigkeit.
  \item Ändern Sie Ihre Clustersuche so ab, dass Sie auch die Größe des größten Clusters bestimmen.
    Der Anteil am gesamten Gitter, der durch den größten Cluster belegt wird, wird mit $M_∞$ bezeichnet.
    Es gilt dabei
    \begin{equation}
      M_∞ (p) \sim |p - p_c|^{-β},
    \end{equation}
    mit einem Exponenten $β$.
    Versuchen Sie $β$ durch Messung von $M_∞(p)$ in Systemen einer möglichst großen Größe $l$ sowie durch Finite-Size-Scaling
    \begin{equation}
      M_∞(p) l^{β/ν} \sim f\left( (p - p_c)^{1/ν} l \right)
    \end{equation}
    zu bestimmen.
    Variieren Sie dazu $β$ und $ν$ so, dass die Datenpunkten möglichst gut auf eine Funktion $f$ kollabieren (wiederum min.\ für $l = 10, 50, 100$).
  \end{enumerate}
\end{question}

\begin{question}[subtitle=Union-Find-Algorithmus]
  Implementieren Sie den Hoshen--Kopelman-Algorithmus nach dem Prinzip des Union-Find für die Cluster-Suche.

  \url{https://www.ocf.berkeley.edu/~fricke/projects/hoshenkopelman/hoshenkopelman.html} (Stand: 28.\ Juni 2019).
\end{question}
\end{document}